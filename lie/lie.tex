\documentclass[a4paper, 12pt, leqno]{report}

\usepackage[french]{babel}
\usepackage[latin1]{inputenc}
\usepackage[T1]{fontenc}
\usepackage{amsmath}
\usepackage{amssymb}
\usepackage{geometry}
\usepackage{fancyhdr}
\usepackage{graphicx}


\title{\huge{Groupes et algebre de Lie}}

\lhead{Groupes et algebre de Lie}
\chead{}
\rhead{EPITA 2008}

\geometry{vmargin=3cm, hmargin=3cm}

% \renewcommand{\H}{\stackrel{\wedge}{H}}
\newcommand{\N}{\ensuremath{\mathbb{N}}}


\pagestyle{fancy}


\begin{document}

\maketitle
\newpage
\tableofcontents
\newpage

\chapter{Introduction}

\section{Rappel sur les groupes}

\paragraph{Definition}

On appelle commutateur de $x$ et $y$, l'element $xyx^{-1}y^{-1}$. On
note $G'$ et on l'appelle groupe derive l'ensemble
$$G' = \left \{ xyx^{-1}y^{-1} \; | \; (x,y) \in G^2 \right \}$$

\paragraph{Propriete}

$G'$ est un groupe, et un sous groupe de $G$.

\paragraph{Definition}

Un groupe $G$ est dit resoluble si il existe $p \in \N^*$ tel que
$G^{(p)} = \left \{ e \right \}$.
\begin{itemize}
\item $G^{(0)} = G$
\item $G^{(p+1)} = \left( G^{(p)} \right)'$
\end{itemize}

\paragraph{Definition}

$G$ est abeblien lorsque $\star$ est commutative. En consequence, tout
sous groupe est distingue ($xHx^{-1} = Hxx^{-1} = H$).

\paragraph{Definition}

Le centre de $G$, note $Z(G)$, est l'ensemble des elements qui
commutent avec tout le monde.

$$Z(G) = \left \{ g \in G \; | \; \forall x \in G \; g \star x = x \star
  g \right \}$$

\end{document}
