Vaucanson est une plateforme logicielle d�di�e � la manipulation
d'automates et de transducteurs. D�velopp� au LRDE en partenariat avec
Jacques Sakarovitch (ENST) et Sylvain Lombardy (LIAFA), Vaucanson est
une biblioth�que g�n�rique permettant d'utiliser et d'�crire des
algorithmes pour une grande vari�t� d'automates.

A la fin de l'�t� 2006, la version 1.0 de Vaucanson est sortie
proposant une suite d'outils en ligne de commande comme interface
utilisateur. La plateforme Vaucanson est d�sormais utilisable sans
mettre en oeuvre de programmation.

Par ailleurs, un prototype d'interface graphique pour Vaucanson a �t�
r�alis� 2004 par Louis-No�l Pouchet. Le but de ce stage est de
produire, � partir d'un prototype existant, une version toujours
exp�rimentale mais diffusable et document�e.

Le logiciel produit devra permettre de visualiser des automates
simples, de modifier sa forme et son apparence graphique et
d'appliquer les algorithmes propos�s par le TAF-KIT � ces automates.
Une grande partie de tout cela est d�j� accompli par le prototype,
cependant une r�organisation du code, la production d'une
documentation d�veloppeur reste � faire, afin de faciliter les
am�liorations futures.
