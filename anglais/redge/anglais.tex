\documentclass[a4paper, 12pt]{article}

\usepackage[english]{babel}
\usepackage[latin1]{inputenc}
\usepackage[T1]{fontenc}

\title{English: world of work}
\author{Guillaume Leroi}

\begin{document}

\maketitle

\section{Working at the LRDE}

I'm working at the LRDE on the Vaucanson project since
2006. Vaucanson is a framework dedicated to the manipulation of finite
state machines. The goal of this project is to attain two objectives:
genericity and performance. Genericity so that Vaucanson can be use on
a wide variety of problems. Performance so that Vaucanson can be use
to solve the hardest problem.

My work at the LRDE consists in maintaining and adding new capacities
to Vaucanson. I take especially an interest in understanding and
programming new algorithms. This year, my particular purpose is to
extend Vaucanson to allow our users to solve problems and use
algorithms from the number theory.

Since January, the LRDE has recruited new students, so three new
students are working now on Vaucanson. An other part of my job at the
laboratory is to help them. With Guillaume Lazarra, we have to guide
them, and give them the basics to work on Vaucanson.

Working at the LRDE is a very motivating experience. As describe
before, I have the occasion in my every day life to think about and
try to solve problems.

One thing that I particularly enjoy, is the wide variety of
problems. We may have to solve optimisations problems, check the
validity of a algorithm\ldots

I like too the lecture given, about mathematics or scientific
computations, and all discussions about how solve them, finding the
balance between efficiency, and available resources.


\section{Epita's pool}

The first two weeks at Epita are called \emph{the pool}. This is a
period where every students must be at school for learning basics of
computer programming. This is a very particular period of a Epita's
student life. Some assistants, called ACU (third year student), are
supervising us, giving us the good practises of computer
programming. The schedule is very loaded, as in two weeks we learn
things that are teach in six months in other schools.

This is a very challenging period for first year students. We all
worked from noon to six o'clock in the morning, every day, even the
week end.
Among computer programming, assistants teach us strictness too. The
grading is very harsh, no mistakes are allowed.

Despite all this difficulties, this was too a very motivating period
of my life. And even if difficulties can get you down easily, the
surrounding keeps you up.
In fact, I enjoyed the surrounding of the pool. The challenge was
motivating us. Working with my friends all night long and facing the
same difficulty creates between us connections, friendship in
adversity.

It was a very rewarding experience. I learned some things on my self:
the load of work I can bear, the joy I can found in computer
programming, the importance of friendship in adversity.


\section{Interviews}

I have interview two friends. They particulary know me well as they
work a lot with in harsh conditions.

\subsection*{Benjamin Ratier}

\begin{verbatim}
<leroi_g>Alors on commence, mes qualites? :)
<agho@jabber.fr> En qualite
<agho@jabber.fr> Tu as l'esprit mathematique
<agho@jabber.fr> Tu es curieux (tu installes emacs serveur, etc)
<agho@jabber.fr> Tu respectes tes ideaux (le libre, les hippies et
tout ca)
<agho@jabber.fr> Franc, simple (ne se prend pas la tete)
<agho@jabber.fr> Tu travailles quand ca te plait
<leroi_g>Et mes d�fauts?
<agho@jabber.fr> En d�faults
<agho@jabber.fr> simple (ne se prend pas la tete), ce qui te fait
parfois oublier des petits d�tails qui font chuter tout le projet
(une compilatrion sous bsd, un format de fichier)
<agho@jabber.fr> Tu noteras l'utilisation du a la fois une qualit�,
 a la fois un d�faut
<agho@jabber.fr> Ca plait les doubles sens !
<agho@jabber.fr> Ne fait pas ce qui te plait pas.
<agho@jabber.fr> Refuse de travailler avec des outils propri�taires,
putain de hippie
<leroi_g> Et mon comportement en groupe?
<agho@jabber.fr> Gloablement c'est agr�able de travailler avec toi
<agho@jabber.fr> Bonne ambiance de travail
\end{verbatim}


\subsection*{Ronan Monfort}

\linespread{0.8}
\begin{verbatim}
Cher Guillaume,
tout d'abord, tu es une des personnes les plus comp�tente en
informatique, qu'il m'ait �t� de rencontrer. En effet, tu as une
culture informatique vaste et vari�e sur de multiples technologies que
ce soit les plateformes, les langages et les int�ractions possibles
entre elles.
Ensuite, tu es de nature discr�te, ce qui te permet de rester tr�s
humble malgr� tes bons r�sultats. C'est une grande qualit� qui est
assez rare � EPITA...
Tu as une grosse volont� de r�ussir et tu fais tout pour arriver aux
objectifs que tu t'es fix�.
Tu restes fid�le � ta philosophie de l'informatique (Free software ==
Gnu Addict).
Tu n'as jamais h�siter � r�pondre � mes questions et � cellles de
nombreux coll�gues qui te sollicitent quasiment tous les jours. Tu
prends en plus le temps d'expliquer correctement pour �tre s�r que la
personne a compris.

Mais pourtant, tu as quelques d�fauts, enfin je n'en vois que deux:
- Tu ne mets pas assez en avant ton travail et cela te d�servira dans
le travail, car ton patron verra m�me pas que tu lui as impl�menter un
algorithme super puissant.
Alors que le petit d�veloppeur d�brouillard recueillera les m�mes
compliments....
- Tu as tendance � ne pas terminer quelques projets par flemme alors
que tu aurais le temps et les capacit�s de le faire...

A part, ces deux petits points que j'ai not� � la fin, tu es pour moi
une des personnes sur lequel je peux compter. De plus, je commence �
te conna�tre plus sur le plan personnel, et je pense qu'on passera
quelques bonnes soir�es ensemble autour d' un verre bien s�r...
\end{verbatim}


\end{document}
