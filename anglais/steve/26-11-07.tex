\documentclass[a4paper, 11pt]{article}

\usepackage[english]{babel}
\usepackage[T1]{fontenc}
\usepackage{geometry}

\geometry{hmargin=2cm, vmargin=2cm}

\title{500 words about my work at the LRDE}
\author{Guillaume Leroi}

\pagestyle{empty}
\linespread{1.5}

\begin{document}
\maketitle
\thispagestyle{empty}

Since September, my work is divided between two main objectives. One the one
hand, the Vaucanson group aims at distributing a new version of its platform, on
the other hand I had to work on my seminar.

\paragraph{}

Preparing the release of a new version of a project is a long and tedious
work. The package has to be clean, \emph{i.e} everything in the package has to
work properly. We have to check that all necessary files are present.

My work was to verify that all our programs work properly. If some of them are
incorrect, I have to check that the user is warned of that somewhere, or to
remove the program.

\paragraph{}

My seminar's subject is synchronized transducers, again. So i re-develop the
synchronization algorithm. I am also working on developing a new data structure
for synchronized transducer.

One of the particularity of a synchronized transducer is that it can be seen as
a classic automaton. Therefore it may be used with already existing algorithms
of Vaucanson for classical automata.

For that reason, a new structure is added to Vaucanson, allowing a synchronized
transducer to be used as an automaton. This addition is bring to light some
problems in Vaucanson and its supposed genericity.

First, Vaucanson is generally used with letters. For synchronized transducers to
be seen as automata, pairs of letter are need. So some compilations and
input/output issues were revealed and need to fixed.

Second, in Vaucanson, every transitions have the same type of label. But
synchronized transducer have two types of label. Therefore, I am extending
Vaucanson to be able to used different types of label in the same
automaton. This with the work of Guillaume Lazzara is a good test for the
genericity of Vaucanson and its extensibility.

\paragraph{}

Finally after months of work, i worked on the graphical user-interface of
Vaucanson. The Vaucanson group has connection with Taiwanese researchers who are
specialized in displaying automata on a screen, and are interested in using
their own automaton's data in Vaucanson.
The user interface is now up-to-date with the last release of Vaucanson.

\end{document}
